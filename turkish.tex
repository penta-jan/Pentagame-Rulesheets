\selectlanguage{turkish}
\renewcommand{\headline}{Penta$\cdot$game - Türkçe}
\renewcommand{\tocent}{Türkçe}
\renewcommand{\translator}{Ayberk Karaaslan}
\renewcommand{\general}{
\lettrine[lines=3]{B}{} ir dakikada anlatılabilen ama büyüleyiciliğini yıllar boyu koruyan bir oyun. 
İki oyuncu ile sadece 20 dakikada oynanabiliyor. Üç veya dört oyuncu ile 90 dakikaya kadar çıkabilir. Zar kullanılmaz. 
5 yaş ve üzeri herkese uygundur. Kutunun içerisinde: 4×5 renkli figür, 5 adet siyah ve 5 adet gri blok ve oyun tahtası bulunmaktadır.
}

\renewcommand{\choosext}{Figürlerinizi seçin}
\renewcommand{\choosex}{
Her oyuncunun beş figürü vardır. Kutuda oyuncu başına bir takım figürler bulunur. Mavi, kırmızı, beyaz, yeşil ve sarı bir figürünüz var. 
Bunlar, tahtanın kendi renkleriyle eşleşen beş köşesinden oyuna başlayarak ortadaki kendi renkleriyle eşleşen çıkış yapabilecekleri duraklara varmak istiyorlar.
}
\renewcommand{\setupt}{Kurulum}
\renewcommand{\setup}{
Figürlerinizi halkadaki büyük köşelere, vücut renklerine uygun şekilde yerleştirin: beyaz figürünüzü halkadaki beyaz köşeye, mavi figürünüzü halkadaki mavi köşeye vb.

Siyah blokları, tahtanın ortasındaki beş adet geçişe yerleştirin. 

Gri blokları daha sonra kullanmak üzere ortaya park edin. 
}

\renewcommand{\objectivet}{Amaç}
\renewcommand{\objective}{
Beyaz figürler merkezdeki beyaz geçişe, mavi olanlar mavi geçişe gitmek isterler, vb. 
Hedef her zaman başlangıç noktasının karşısındaki orta beşgendeki renkli büyük duraktır.

Kazanmak için \emph{üç} adet figürü hedeflerine taşıyın.
}

\renewcommand{\rulest}{Kurallar}
\renewcommand{\rules}{
Figürlerinizden herhangi birini yıldızın veya halkanın üzerinde istediğiniz yöne doğru istediğiniz kadar hareket ettirin. 

Herhangi bir boş köşeden veya kesişim noktasından duraksama yapmadan dönebilirsiniz.

Asla ne blokların, ne de herhangi bir figürün üzerinden atlamayın. 

\myskip

Eğer aşağıdakilerden birisini yaptıysanız işgal edilmiş bir durağa geçebilirsiniz:

\myskip

Siyah bir bloğu kırdınız ve onu seçtiğiniz serbest bir durağa yerleştirdiniz.

Başka bir figürün olduğu durağa gitmek istiyorsunuz, bu sebeple onunla konumlarınızı değiş-tokuş ettiniz.
 
\myskip

Kendi figürlerinizden ikisinin konumunu birbirleri arasında değiş-tokuş ettiniz. 

Birden fazla figürün bulunduğu bir durağa hareket ederseniz, konum değişimi için birisini seçin.
 


\myskip 
 
Aynı hareketi art arda iki kez yapmayın.  

\myskip 

Hedefine ulaşan figür kaldırılır ve tahtanın ortasına konulur. Daha sonra gri bloklardan birini alın ve dilediğiniz bir serbest durağa koyun.

Eğer böyle gri bir bloğu kırarsanız, onu tahtadan tekrar alın.

Gri bloklar biterse oyunda zaten bulunmakta olan bloklardan birini yeniden konumlandırın.

\myskip 

Başka bir oyuncu sizi hedefinize doğru hareket ettirdiyse sıra size geldiğinde oradan \emph{ayrılmalısınız}

\myskip

Son tur oynanana kadar bu düzende devam edilir.
% Her kim elindeki \emph{üç} figürü hedeflerine ilk olarak ulaştırmayı başarırsa o kazanır. %
}