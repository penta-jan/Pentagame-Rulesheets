%\selectlanguage{nederlands}
\renewcommand{\headline}{Penta$\cdot$spel - Nederlands}
\renewcommand{\tocent}{Nederlands}
\renewcommand{\translator}{Daniel Franke}
\renewcommand{\general}{
    \lettrine[lines=3]{E}{en} spel dat men in een minuut snapt, maar dat jarenlang blijft fascineren.
    Met twee spelers duurt een spel 20 minuten. Met drie of vier spelers duurt het 40 tot 90 minuten.
    Het spel is geschikt voor iedereen vanaf 5 Jaar en wordt zonder dobbelstenen gespeeld.
    In de doos zitten 4$\times$5 gekleurde figuren, 5 zwarte blokkades, 5 grijze blokkades en het spelbord.
}

\renewcommand{\choosext}{Figuren}
\renewcommand{\choosex}{
    Elke speler heeft een team van vijf figuren van een vorm naar keuze.

    In elk team zit een rood, blauw, wit, groen en geel figuur.

    Elk figuur begint op de hoek van de eigen kleur, en eindigt op de kruising van de eigen kleur, waar het van het bord verwijderd wordt.
}

\renewcommand{\setupt}{Opbouw}
\renewcommand{\setup}{
    Zet je figuren op de hoeken van dezelfde kleur, het witte figuur op de witte hoek, het gele figuur op de gele hoek, enz.
    
    Zet de zwarte blokkades op de gekleurde kruisingen in het midden.

    Zet de grijze blokkades voorlopig in de open ruimte in het midden van het bord.
}

\renewcommand{\objectivet}{Doel van het spel}
\renewcommand{\objective}{
    Witte figuren moeten naar de witte kruising, gele figuren naar de gele kruising, enz.
    Het doel is altijd de kruising van dezelfde kleur tegenover het beginpunt.

    De winnaar is degene die het eerst \emph{drie} figuren uit het spel krijgt.
}

\renewcommand{\rulest}{Spelregels}
\renewcommand{\rules}{
    Verplaats per beurt een van je eigen figuren in elke richting, zo ver als je wilt.
    
    Je kan bij elke hoek of kruising van richting veranderen zonder de zet te beëindigen.

    Je kan hierbij nooit over andere figuren springen, dus ook niet over blokkades.

    \myskip

    Maar je mag met je figuur wel slaan door het \emph{op} een bezet veld te zetten:

    \myskip

    Als je een zwarte blokkade slaat, mag je de blokkade naar een willekeurig vrij veld verplaatsen. 

    Als je een ander figuur slaat, ruil je van positie met dat figuur.

    \myskip

    Je kan op deze manier dus ook twee van je eigen figuren van plek verwisselen.

    Wanneer je naar een veld met meerdere figuren zet, sla je maar \emph{één} van die figuren.

    \myskip 
 
    Je mag niet twee keer achter elkaar dezelfde zet doen.

    \myskip 

    Een figuur dat zijn doel heeft bereikt gaat uit het spel. Deze wordt in het midden van het bord gezet. Je kan nu een van de grijze blokkades op een vrij veld van het bord zetten.

    Als er geen grijze blokkades meer in het midden staan, verplaats je een grijze blokkade die al op het bord staat.

    Wanneer een grijze blokkade wordt geslagen, wordt deze weer uit het spel verwijderd en naar het midden verplaatst.

    \myskip 

    Wanneer een andere speler een van jouw figuren naar zijn doel heeft verplaatst, \emph{moet} je deze verwijderen in je volgende beurt.

    \myskip

    De laatste ronde gaat tot het einde door zodat iedereen evenveel zetten heeft gespeeld.
    Als het meerdere spelers lukt om 3 figuren uit te spelen, is het gelijkspel. 
}
