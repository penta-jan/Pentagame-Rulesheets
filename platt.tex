\selectlanguage{ngerman}
\renewcommand{\headline}{Penta$\cdot$speel - Plattdüütsch}
\renewcommand{\tocent}{Plattdüütsch}
\renewcommand{\translator}{J.S.}
\renewcommand{\general}{
In een Minuut to begriepen liekers plietsch noog johrelong to speeln.

Twee Speler broken blots 20 Minuten. Dree orn veer speeln 40-99 Minuten.

Wörpel gifft dat nich. Heel goot för Speler af 5 Johren. En Speel vun Jan \noun{Suchanek}.

In de Lood: 4$\times$5 farvige Poppen, 5 swatte un 5 grooge Blöck un de Speelbrett.
}

\renewcommand{\choosext}{Poppen}
\renewcommand{\choosex}{
Elk Speler föhrt en Koppel vun fief Poppen.

Jedereen söökt en so'n Koppel ut. 

In jede Koppel is 'ne blage, 'ne roote, 'ne witte, 'ne grööne un 'ne geele Popp. 

De trecken all vun de fief Ecken to de fief Krüüzen. 
}

\renewcommand{\setupt}{Opbuu}
\renewcommand{\setup}{

Stellt de Poppen no Forben ordnet op de fief Ecken op de Krink: de witten op de witte Eck, de blage op de blage, usw. 

Op jedereen Krüüz kummt en swatte Block.

De groogen bleven eerstmal in de Mitt. 
}

\renewcommand{\objectivet}{Teel vun de Speel}
\renewcommand{\objective}{
Witte Poppen wulln to dat witte Krüüz, blage to dat blage, usw. 

Vun de Krink her is de Teel ümmer de Krüüz mit de sülve Farve gegenöver. 

Welkeen toeerst \emph{dree} vun sünne Poppen op de rechte Teel treckt, winnt.
}

\renewcommand{\rulest}{Regelwark}
\renewcommand{\rules}{
Treck een vun diene Puppen op de Stiern or de Krink as de möögst in egal welke Richtung. 

An jedereen Krüüz kannst afbögen ahn anhoollen. 

Du dörvst trecken, liekers nich jumpen! Kien över Blöck of över anner Poppen!

\myskip

Aver du kannst op een Flach trecken, dat besett is, un slaan:

\myskip

Slaagst du en swatten Block, sett eem up egal en frieet Flach. 

Slaagst du en anner Popp, denn tuuschen de beeden Poppen de Steel.

\myskip

So kann man ook twee vun sünne egenen Poppen tuuschen.

Treckst du op een Flach, op de noch veele Poppen stahn, muttst mit \emph{een }vun jüm tuuschen. 

\myskip 

Man dörv nich tweemol de sülven Treck maken. 

\myskip 

Kummt  en Popp to ehr Teel hen, treckt se rut. Se kömmt in de Mitt. Dorför kreegt man vun dor een groogen Block op en freet Flach to setten. 

Slaagt man so en groogen Block, kummt he wedder rut.

\myskip

Welkeen toeerst \emph{dree }vun sien Poppen uttreckt, winnt. 

\myskip

Wenn du nich ran büst, hool din Snuut!
}
