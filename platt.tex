\renewcommand{\headline}{Penta$\cdot$speel - Plattdüütsch}
\renewcommand{\tocent}{Plattdüütsch}
\renewcommand{\translator}{J.S.}
\renewcommand{\general}{
Kannst in een Minuut spitzkriegen liekers bleevt johrelang smakelk.

Twee Speler broken blots 20 Minuten. Dree or veer speeln 40-99 Minuten.

Wörpel gifft dat nich. Passlich för Speler af 5 Johren. En Speel vun Jan \noun{Suchanek}.

In de Lood: 4$\times$5 farvige Poppen, 5 swatte un 5 griese Blöck un de Speelbrett.
}

\renewcommand{\choosext}{Poppen}
\renewcommand{\choosex}{
Elk Speler föhrt en Koppel vun fief Poppen.

Jedereen söökt en so'n Koppel ut. 

In jede Koppel is 'ne blage, 'ne roote, 'ne witte, 'ne grööne un 'ne geele Popp. 

De trekken all vun de fief Ecken to de fief Krüüzen. 
}

\renewcommand{\setupt}{Opbuu}
\renewcommand{\setup}{

Stellt de Poppen no Forben ordnet op de fief Ecken op de Krink: de witten op de witte Eck, de blage op de blaage, usw. 

Op jedereen Krüüz kummt en swatte Block.

De griesen bleven iersmol in de Midd. 
}

\renewcommand{\objectivet}{Teel vun de Speel}
\renewcommand{\objective}{
Witte Poppen wulln to dat witte Krüüz, blaage to dat blaage, usw. 

Vun de Krink her is de Teel ümmer de Krüüz mit de sülve Farve op de günt siet. 

Welkeen toeerst \emph{dree} vun sünne Poppen op de rechte Teel trekkt, winnt.
}

\renewcommand{\rulest}{Regelwark}
\renewcommand{\rules}{
Treck een vun diene Puppen op de Stiern or de Krink as de möögst in elke Richt. 

An jedereen Krüüz kannst afbögen ahn stoppen. 

Du dörvst trekken, liekers nich jumpen! Kien över Blöck of över anner Poppen!

\myskip

Aver du kannst op een Feld trekken, dat besett is, un slaan:

\myskip

Slaagst du en swatten Block, sett eem up jichens en frieet Feld. 

Slaagst du en anner Popp, denn tuuschen de beeden Poppen de Steel.

\myskip

So kann man ook twee vun sünne egenen Poppen tuuschen.

Trekkst du op een Feld, op de noch veele Poppen stahn, muttst mit \emph{een }vun jüm tuuschen. 

\myskip 

Man dörv nich tweemol de sülven Treck maken. 

\myskip 

Kummt  en Popp to ehr Teel hen, trekkt se rut. Se kömmt in de Midd. Dorför kreegt man vun dor een griesen Block op en freet Feld to setten. 

Slaagt man so en griesen Block, kummt he wedder rut.

\myskip

Welkeen toeerst \emph{dree }vun sien Poppen uttrekkt, winnt. 

\myskip

Holl di stiev un sabbel keen dumm Tüüch!
}
